% Document format
\documentclass[11pt]{article}
% Paper size and margins
\usepackage[a4paper,left=2.25cm,right=2.25cm,top=2cm,bottom=2.5cm]{geometry}
% Professional-looking tables
\usepackage{booktabs}
\usepackage{threeparttable}
% APA style citations and references
\usepackage{apacite}
% Nice web addresses
\usepackage{url}
% Allow whole sections to be commented out easily
\usepackage{comment}

\begin{document}
\title{PSYC403: Aspects of Consciousness - Essay Guidance}

\author{Andy J. Wills}
\date{\today}
\maketitle


\section{Summary}
Over the next few pages, I give quite a lot of detail about what I expect from the essay, and how to approach researching and writing your essay. If you follow this advice fully and successfully, it is likely you will get a good mark.

\begin{flushright}
	\emph{Andy J. Wills}
\end{flushright}

\section{What are the questions?}

If you are submitting your essay at the normal time, you may choose between one of the two following questions. If the exam baord has given you special dispensation to answer this essay over the summer (known as a referred assessment), \textbf{the questions are different to the ones shown below}. Do not answer these questions if you are doing a referred assessment.

\begin{enumerate}

\item What does the neuroscience and psychology of (a) voluntary action, and (b) drug addiction, tell us about free will?

\item Critically evaluate what at least two of the following three research areas have told us about consciousness: (1) perception, (2) attention, (3) recollection. 

\end{enumerate}


\section{Where to start?}

The essay titles are rooted in the lecture material, but you are expected to go further than the lecture material. Start with the relevant references in the course guide, and then use these to find more research. Do this by (a) following up relevant studies in their reference sections, (b) doing a cited reference search on these papers (use Web of Science for this). 

You'll likely end up with a vast number of papers. So, narrow this down on four criteria:

\paragraph{Citations} - If a piece of work has been cited a lot, this means it has had a significant impact on the field (not necessarily a positive impact). For the purposes of this essay, you can safely ignore articles cited fewer than 10 times (unless recent, see below).

\paragraph{Recency} - The impact of recently-published work is as yet unknown; it is therefore unwise to use citations to select recently published articles. For the purposes of this essay, consider all articles published on or after 1st January 2012 to be recent. Including some recent work in your essay is important.

\paragraph{Relevance} - Will this article help you answer the question set? Use the title and, if necessary, the abstract, to work this out. You'll need to use your own judgement to do this -- this is a key skill and part of the assessment.

\paragraph{Availability} - The library's electronic resources are good, but not infinite. If the article you've found is not in the library, try searching more generally on the internet - for example, many authors provide copies of their papers on personal websites. You could also email authors to ask for a copy if it's not available in the library.

\section{How to priortise your reading}

You may not have the time to read all of the articles you now have, and you should certainly prioritise your reading, whether you intend to read them all in the end or not. My advice is that you read reviews and grand reviews first. This will give you a good understanding of what the phenomena are. Then read more specific, empirical papers. 

\paragraph{Grand reviews} - Review articles that incorporate your chosen topic as one of several phenomena to make certain theoretical points. Such articles can be useful to put your topic a broader context, and may help answer the consciousness aspect of the essay. Read these strategically to extract the components relevant to your essay.

\paragraph{Topic reviews} - Review articles that integrate across several previous reports on your topic. 

\paragraph{Empirical papers} - Articles that give detail on a specific, or short series, of experiments. An understanding of methodological detail and results is important to show in your essay. 

\section{How to write a good essay}

\paragraph{Total time hypothesis} - The best predictor of how much you learn, retain, and understand is how much time you spend deliberately studying the material, giving it your full attention during periods of study. 

\paragraph{Space your learning} - Do not attempt to read everything in 1-2 days. Put aside some time every week from now to the deadline to read and write for this essay. Sleep consolidates learning.

\paragraph{Answer the question} - It is absolutely critical that you answer the question in full, and that it is obvious that you have done so. Essays not fully addressing the question will receive low marks. 

\paragraph{Quality of understanding} - I strongly recommend trying to summarize an article in note form immediately after you have read it, referring back to the article as little as possible whilst you do this. Then check your summary against the article for accuracy. Although your essay must be written independently, there is no law against discussing the topic with your friends and colleagues. Indeed, I strongly encourage doing this, as it will improve your understanding and theirs.

\paragraph{Evidence-based argument} - use it! It is not enough to summarize the conclusions of others, you must provide a sense of the evidence that supports those conclusions, and how the conclusions follow from the evidence.

\paragraph{Quality of evidence} - Read what you like on the internet to aid your introduction to this area. There are likely a number of layperson introductions to your topic out there. However, be aware that anyone can post stuff on the internet, and just because it's there doesn't mean it's true, and it certainly doesn't mean you can use it in your reference section. You reference section should \emph{only} contain peer-reviewed journal articles. Do not cite books, book chapters, and so forth (although again, if you come across something useful, by all means read it).

\paragraph{Quality of structure} - Plan your essay before you write it. This means writing a list of bullet points, getting the structure right, and then fleshing out the structure from there. Before starting to write, look at your structure. Does it follow a logical progression, like a good story? Or does it flit back and forth, returning to essentially the same points several times? If the latter, re-order to minimize this. Essay structure has rules; these rules are basically those set out by the Ancient Greeks, and not much has changed since then, see \citeA{rh85}. There is a good brief summary of this classic work on wikipedia.

\paragraph{Appropriate Audience} - Although I mark the essay, your audience is \emph{not} me. Your audience is essentially a Stage 2 psychology student. If you're talking about a concept or technical term that you learned in Stage 1 or 2, it is not essential to define it. If it is a concept or term you learned in Stage 4, define it before using it.

\paragraph{Quality of Presentation} - Make sure your essay reads well, with well-constructed sentences, and an absence of grammatical problems. Make sure you follow APA style slavishly. It you use acronyms, define them before using them.

\section{Last year's essays}

Here are the main things that the weaker essays last year had in common; essays with two or more of these problems largely failed to reach a 2i
standard: 

\paragraph{Not going beyond the lecture material} It is OK to cover
some of the material I directly taught you in class, but (as covered
in the guidance notes) you must go beyond this in a meaningful way to
score well.

\paragraph{Absence of critical evaluation of the presented material}
What are the alternative explanations? Any methodological problems?
What could be done differently / better / in addition to the reported
study?  Often, more recent papers that follow up (and hence cite)
well-known studies are a great source of such information.

\paragraph{Absence of evidence-based argumentation} (i.e. the
presentation of specific evidence, with some sense of the method and
results, and how those results lead (or fail to lead...) to the
conclusions drawn by the authors.

\paragraph{Too much general text} - not infrequently, poorer essays
used up about 2 pages at the beginning making very general comments
that do not directly answer the question, and they use up the last
page saying what they had already said more briefly (a conclusion
concludes - it draws together your thesis - it is not just a miniature
form of the preceding essay).

\section{A minor point}

The Stroop task is probably one of the most famous tasks in
psychology. It was invented by John Ridley Stroop. So it's the Stroop
test, not the 'stroop' test, as perhaps the majority of people last
year wrote in their essay. This error implies to the reader either
that you don't know Stroop was a person, or that you don't know the
difference between proper and improper nouns. Of course, neither is a
good impression to give in a degree-level psychology essay.

\bibliographystyle{apacite}
\bibliography{masterbib3} {}

\end{document}

%%% Local Variables:
%%% mode: latex
%%% TeX-master: t
%%% End:

% Document format
\documentclass[11pt]{article}
% Paper size and margins
\usepackage[a4paper,left=2.25cm,right=2.25cm,top=2cm,bottom=2.5cm]{geometry}
% Professional-looking tables
\usepackage{booktabs}
\usepackage{threeparttable}
% APA style citations and references
\usepackage{apacite}
% Nice web addresses
\usepackage{url}

\begin{document}
\title{Aspects of Consciousness}
\author{Andy J. Wills}
\date{\today}
\maketitle

\section{Before we begin...}

The readings for this series of classes are primarily journal
articles. If you'd like to read something at a slightly more
introductory level before diving into these primary sources, take a
look at \emph{Concsciousness: An Introduction}
\cite{blackmore2010}. This textbook is a good general introduction to
the study of consciousness.

\subsection{A note on slides, notes, and recordings}

It is a basic principle of open science that all information should be
stored in open formats. This ensures that access is not restricted to
those who own a particular piece of proprietary software. The file
formats I use in these seminars are PDF (for slides and notes) and OGG
(for audio recordings). The free and open source Firefox web browser
supports both these formats. Alternatively, look here for information
on other options for opening these files:

\emph{PDF:} \url{https://en.wikipedia.org/wiki/List_of_PDF_software#Viewers}

\emph{OGG:} \url{https://en.wikipedia.org/wiki/Wikipedia:Media_help_%28audio_and_video%29}

Except where noted, all materials for these seminars are licensed
under a Creative Commons Attribution-ShareAlike 4.0 International
Licence.

\section{Introduction}

Consciousness is a fascinating but, at first sight, intractable
topic. A central theme of this series of classes is that
consciousness, like other vague terms (e.g. emotion, memory, tyranny),
is not possible to study directly--one must instead define and
consider somewhat more specific questions. The hope and expectation is
that, through defining and answering enough specific questions that
fall under the umbrella term of consciousness, our understanding of
consciousness as a whole will improve.  It is possible that, by so
doing, we will end up discarding the term consciousness entirely, in
the same way that every secondary school kid has a reasonable grasp of
the concept of \emph{oxygen} but only historians of science talk about
the once-popular concept of \emph{phlogiston} \cite{phlogiston}.

This course will not provide any simple explanation of consciousness;
its aim is to engage you in the debates--some empirical, some
philosophical--that define this field of study. At the end of the
course, I hope you will have a clearer and deeper understanding of the
issue, and how progress has been made in studying consciousness in a
scientific manner. Across 5 two-hour classes, we will explore the
following topics: (1) Conceptions of consciousness, (2) Consciousness
and perceptual awareness, (3) Altered states of consciousness, (4)
Conscious recollection, (5) Conscious control.

I hope you will find engaging with this material as interesting, and
as challenging, as I found it to put together.

\begin{flushright}
	\emph{Andy J. Wills}
\end{flushright}

\section{What should you be reading?}

The classes are designed to be an introduction to the material
contained in the References. You should also be reading extensively,
and you should base your reading around the Reference section of this
document. You should set aside about 20 hours per lecture for reading,
understanding, and making notes on these References. Make as many
connections as you can between what you have just read, and everything
else you have read already for this topic. The exercise of doing this
will improve your understanding of the
material. 

In some cases, I reference more than 20 hours of material per class,
so you may have to prioritise; as final-year undergraduates you should
be up to this challenge. Note that some of the References are books or
wikipedia articles; secondary sources like these are good for breadth,
but your anwers to assessment questions should use only primary
sources.


\section{Course structure}

\subsection{Conceptions of consciousness}

We start with a discussion of what the term \emph{consciousness} means
to each of us. I then introduce some key ideas and terms in the study
of consciousness, including the distinction between P-consciousness,
and A-consciousness \cite{block95}, and the nature of free will
\cite{freewill}.  We then tackle the nature of free will, specifically
voluntary action, from a neuroscience perspective
\cite{lau04,libet83,haggard05}.

\subsection{Consciousness and perceptual awareness}

I start with some demonstrations of some compelling illusions that
serve to illustrate that conscious awareness is narrower
\cite{simons05}, and less unitary \cite{aglioti95}, than introspection
might suggest. We then consider the implications of \emph{blindsight}
patients \cite{cowey04, azzopardi97, kentridge99} for theories of
conscious awareness. After the break, we consider the existence
\cite{williams, kolb95} of perception without awareness in
non-brain-damaged individuals \cite{merikle01}. Outside the class, I
encourage you to follow up the work of \citeA{fendrich01}, which is
also relevant to the topic of this session.

\subsection{Altered states of consciousness}

We start with a brief review of last week's material and consider the
nature of perception without awareness in non-brain-damaged
individuals \cite{debner94, mccormick97, moore97, marcel83}. We then
move on to a discussion of the meaning and phenomenology of the term
\emph{altered states of consciousness}.  I then review what we know of
the cognitive and neural consequences of meditation \cite{jha07,
  lutz08, lutz04}, and also briefly speak to its therapeutic potential
\cite{miller95, ramel04, tang07}. After the break, we consider what,
if anything, research into 5-HT agonists (e.g. LSD) have revealed
about the nature of human consciousness \cite{nichols04}.

\subsection{Conscious recollection}

We begin with a literary example of \emph{remembrance}
\cite{proust}. I use this example to introduce the concepts of
episodic memory, autobiographical memory, and \emph{autonoetic}
consciousness, and review evidence that an intact hippocampus, whilst
not required for some types of memory retrieval \cite{cermak85}, is
required for the episodic \cite{rempel96} and perhaps the deliberate
\cite{badd94}, use of long-term memory. After the break, we consider
the idea that relatively simple organisms, such as birds, appear to
have episodic memories \cite{clay98}. Such evidence seems to imply
that we must go beyond the definition of episodic memory to articulate
what is special about remembrance. We consider the evidence that the
frontal lobes are the seat of autonoetic consciousness
\cite{wheeler97}, and close with a consideration of evidence that
autobiographical memories may be neurally special \cite{fink99}.

\subsection{Conscious control}

We consider research on dopaminergic drug addiction \cite{robinson03}
and, under the assumption that addiction is a paradigmatic case of the
failure of conscious control, discuss what such research tells us
about human consciousness.

\bibliographystyle{apacite}
\bibliography{masterbib3} {}
\end{document}

%%% Local Variables:
%%% mode: latex
%%% TeX-master: t
%%% End:
